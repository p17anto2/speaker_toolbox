\documentclass[a4paper, 11pt]{article}

\usepackage[english, greek]{babel}
\usepackage{gfsartemisia}
\usepackage[T1]{fontenc}
\usepackage{graphicx}
\graphicspath{ {/media/data/Documents/Panepisthmio/} }
\usepackage{tabularx}
\usepackage{hyperref}
\def\en{\selectlanguage{english}}
\def\gr{\selectlanguage{greek}}
\newcommand{\wen}[1]{\en{}{#1}\gr{}}
\addto\captionsenglish{\renewcommand{\refname}{\gr{}Αναφορές}}

%Use template with ALEFix
%For Greek, \gr
%For English, \en
%For specific words in English, \wen
%To add a good word to the dictionary, use zg and zug to undo
%To add a wrong word to the dictionary, use zw and zuw to undo
%To show suggestions, use z=

\begin{document}
\begingroup
\begin{figure}[t]
	\vspace{-4cm}
	\hspace{-3.5cm}
	\includegraphics[width=3cm, keepaspectratio]{uni.eps}
\end{figure}
\center{}
\LARGE\textbf{Επεξεργασία Ομιλίας και Ήχου: Πρόταση Θέματος}\\
\LARGE\textbf{\wen{Speaker Toolbox}}\\
\vspace{1cm}
\begin{tabularx}{\textwidth}{lXr}
	\hspace{-1cm}\LargeΑντωνακάκης Απόστολος &  & \hspace{3cm}\LargeΠ2017095 \\
    \hspace{-1cm}\LargeΜιχάλης Ζώης (Συγγνώμη αν το σκότωσα) &  & \hspace{3cm}\Large\wen{TODO}
\end{tabularx}
\endgroup

\section{Πεδίο Ενδιαφέροντος}
\paragraph{} \wen{TODO}

\section{Βασικά Στοιχεία Εφαρμογής}
\begin{enumerate}
  \item \textbf{Απομόνωση Επικρατέστερων Συχνοτήτων:} Μέσω του μετασχηματισμού \wen{Fourier}, θα εντοπίζονται 
    οι συχνότητες και αντίστοιχες εντάσεις τους, ώστε να απομονωθούν αυτές που αντιστοιχούν στη φωνή του 
    ομιλητή. Η αναγνώριση των συχνοτήτων θα γίνεται από το χρήστη, ο οποίος θα ορίζει ένα όριο έντασης, κάτω
    από το οποίο οι οι εντάσεις των συχνοτήτων θα αφαιρούνται, ή θα μειώνονται. (Παράδειγμα \wen{Fourier 
    Transformation} στο \wen{Matlab \url{https://www.mathworks.com/help/matlab/math/fourier-transforms.html}})
  \item \textbf{\wen{De-Esser:}} Στις ηχογραφήσεις φωνής, μερικές φορές κάποια σύμφωνα ακούγονται περισσότερα
    από τα υπόλοιπα (\wen{Sibilance} Πηγή: \wen{\url{https://www.sageaudio.com/blog/pre-mastering/sibilance-can-control-vocals.php}}). Οι συχνότητες στις οποίες συμβαίνει αυτό είναι γύρω στα 5--8 \wen{kHz} (Πηγή: 
    \wen{\url{https://theproaudiofiles.com/vocal-sibilance/}}). Με παρόμοια διαδικασία όπως και πριν, ο 
    χρήστης θα μπορεί να εντοπίσει τις συχνότητες αυτές και να τους μειώσει την ένταση.
  \item \textbf{\wen{Low Cut Filter:}} Κόβει χαμηλές συχνότητες.
  \item \textbf{\wen{High Cut Filter:}} Κόβει ψηλές συχνότητες.
\end{enumerate}

\end{document}


