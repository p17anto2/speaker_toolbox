\documentclass[a4paper, 11pt]{article}

\usepackage[english, greek]{babel}
\usepackage{gfsartemisia}
\usepackage[T1]{fontenc}
\usepackage{graphicx}
\graphicspath{ {/media/data/Documents/Panepisthmio/} }
\usepackage{tabularx}
\usepackage{hyperref}
\def\en{\selectlanguage{english}}
\def\gr{\selectlanguage{greek}}
\newcommand{\wen}[1]{\en{}{#1}\gr{}}
\addto\captionsenglish{\renewcommand{\refname}{\gr{}Αναφορές}}

%Use template with ALEFix
%For Greek, \gr
%For English, \en
%For specific words in English, \wen
%To add a good word to the dictionary, use zg and zug to undo
%To add a wrong word to the dictionary, use zw and zuw to undo
%To show suggestions, use z=

\begin{document}
\begingroup
\begin{figure}[t]
	\vspace{-4cm}
	\hspace{-3.5cm}
	\includegraphics[width=3cm, keepaspectratio]{uni.eps}
\end{figure}
\center{}
\LARGE\textbf{Επεξεργασία Ομιλίας και Ήχου: Πρόταση Θέματος}\\
\LARGE\textbf{\wen{Speaker Toolbox}}\\
\vspace{1cm}
\begin{tabularx}{\textwidth}{lXr}
	\hspace{-1cm}\LargeΑντωνακάκης Απόστολος &  & \hspace{5cm}\LargeΠ2017095 \\
    \hspace{-1cm}\LargeΜιχαήλ Ζώης &  & \hspace{5cm}\LargeΠ2010057
\end{tabularx}
\endgroup

\section{Σκοπός Εφαρμογής}
\paragraph{} Το \wen{Speaker Toolbox} θα είναι μία \wen{desktop} εφαρμογή που θα περιέχει μερικές λειτουργίες
καθαρισμού/βελτιστοποίησης ηχογραφήσεων ομιλίας. Σκοπός είναι η τελική εφαρμογή να μπορεί να χρησιμοποιηθεί 
από χρήστες (π.χ. φοιτητές που θέλουν να παρουσιάσουν μια εργασία μέσω
\href{https://www.youtube.com/watch?v=UNVJ3mweSnA}{\wen{Youtube}}) που δεν είναι απαραίτητα εξοικειωμένοι με εφαρμογές επεξεργασίας ήχου (π.χ. \wen{\href{https://www.reaper.fm/}{Reaper},
\href{https://www.audacityteam.org/}{Audacity}}), και δεν έχουν πρόσβαση σε επαγγελματικό εξοπλισμό
ηχογράφησης. Για να επιτευχθεί αυτό, η εφαρμογή θα έχει μερικές απλές λειτουργίες προς αυτό το σκοπό με 
ξεκάθαρες οδηγίες.

\section{Βασικά Στοιχεία Εφαρμογής}
\begin{itemize}
  \item \textbf{Απομόνωση Επικρατέστερων Συχνοτήτων:} Μέσω του μετασχηματισμού \wen{Fourier}, θα εντοπίζονται 
    οι συχνότητες και αντίστοιχες εντάσεις τους, ώστε να απομονωθούν αυτές που αντιστοιχούν στη φωνή του 
    ομιλητή. Η αναγνώριση των συχνοτήτων θα γίνεται από το χρήστη, ο οποίος θα ορίζει ένα όριο έντασης, κάτω
    από το οποίο οι οι εντάσεις των συχνοτήτων θα αφαιρούνται, ή θα μειώνονται.
    (\href{https://www.mathworks.com/help/matlab/math/fourier-transforms.html}{Παράδειγμα \wen{Fourier
    Transformation} στο \wen{Matlab}})
  \item \textbf{Αφαίρεση Συχνών Συχνοτήτων:} Αν μία συχνότητα έχει μεγαλύτερη συχνότητα εμφάνισης από ένα όριο 
    που θα ορίζει ο χρήστης, αυτή η συχνότητα θα θεωρείται θόρυβος και θα αφαιρείται (ή θα μειώνεται η έντασή 
    της).
  \item \textbf{\wen{De-Esser:}} Στις ηχογραφήσεις φωνής, μερικές φορές κάποια σύμφωνα ακούγονται περισσότερα
    από τα υπόλοιπα (\wen{Sibilance~\cite{sibilance}}). Οι συχνότητες στις οποίες συμβαίνει αυτό είναι γύρω
    στα 5--8\wen{kHz}~\cite{sibilance_freq}.  Με παρόμοια διαδικασία όπως και πριν, ο 
    χρήστης θα μπορεί να εντοπίσει τις συχνότητες αυτές και να τους μειώσει την ένταση.
  \item \textbf{\wen{Low Cut Filter:}} Φίλτρο που αφαιρεί σταδιακά συχνότητες κάτω από ένα όριο που 
    ορίζει ο χρήστης.
  \item \textbf{\wen{High Cut Filter:}} Φίλτρο που αφαιρεί σταδιακά συχνότητες πάνω από ένα όριο που ορίζει ο
    χρήστης.
\end{itemize}
 
\section{Παραδοτέα}
\begin{enumerate}
  \item Ανοιχτός Κώδικας στο \wen{Github}.
  \item Αναφορά σε μορφή \wen{PDF} των πεπραγμένων και των επιλογών που έγιναν κατά τη διάρκεια δημιουργίας
    της εφαρμογής.
  \item Βίντεο Παρουσίασης της εφαρμογής.
  \item Παρουσίαση \wen{Powerpoint} της εφαρμογής.
\end{enumerate}

\en{}
\begin{thebibliography}{9}
  \bibitem{sibilance} {\url{https://www.sageaudio.com/blog/pre-mastering/sibilance-can-control-vocals.php}}
  \bibitem{sibilance_freq} \url{https://theproaudiofiles.com/vocal-sibilance/}
\end{thebibliography}
\end{document}


